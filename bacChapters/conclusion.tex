%contents:
%- comparison with related work
%- discussion of open issues

%conclusion: does splitting has to do with complexity with mesh? no

The revelation of occluded objects in order for examination is a complex challenge. In my thesis I presented several approaches that provide different solutions for revealing regions of interest within complex 3D data. Solutions for both volume data and polygonal meshes were adduced with techniques for ghosting, cutaways and exploded views. Furthermore, VolumeShop, a developer tool for visualization research has been described. Its main advantage is the flexible, adaptive and plug-in based concept that lets users develop the optimal solution for their individual needs. Finally, I demonstrated the implementation of a simple algorithm to split meshes in VolumeShop. My implementation of splitting a mesh combined several existing approaches. The cutaway was covered by adjusting a plane to define the prospective cut and combined with a technique similar to an exploded view, implemented with a property in the user interface to adjust the degree-of-explosion. Moreover, an appropriate shading for the revealed structure can be chosen, depending on individual requirements.\\
The main advantage of my algorithm is that regardless of the complexity of the mesh, it has a stable performance as the objectionable fragments are simply discarded in the shader.

TODO add disadvantage: the mesh has to be rendered twice. A discussion on the possible use cases where your method is superior to for instance object-space splitting would be very appropriate.