%contents:
%- motivation
%- problem statement
%- aim of the work
%- methodological approach
%- structure of the work

The main purpose of 3D computergraphics is to represent image data. To make these data examinable, user interaction becomes an important part. Hence, the user should be able to translate, rotate and scale the 3D object. With the complexity of the object, the desire to examine just parts of it rises. The examination and analysis of the objects' inner structures can also be of interest. That leads to the a complex problem that needs to be solved: How can we reveal the inner sturctures? By simply removing the outer structures, the context would also get lost that could be important for scientific findings. Hence, the idea is to reveal the point of interest whilst keeping the context. Therefor, several scientific approaches exist. One idea is to simply raise the opacity of the outer structures. For example to reveal a brain, the approach would be to simply raise the opacity of the scull and the skin respectively for making the parts of secondary interest semi-transparent. Another approach is to cut out parts of the outer structure of the object so that no information gets lost and still the inside would be fully visible. A different approach are so called explosion views that unfold the inside by breaking the outer struktures into multiple parts and shifting those apart whilst keeping the point of interest in the center.\\

This paper will examine several approaches to reveal inner structures of complex 3D models retaining the context to be able to examine the object in respect of its surroundings.\\
TOEDIT In section BLA you will learn how to BLA and BLA BLA