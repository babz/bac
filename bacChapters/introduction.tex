%contents:
%- motivation
%- problem statement
%- aim of the work
%- methodological approach
%- structure of the work


An substantial part of computer graphics is 3D data visualization. To make these data examinable, user interaction plays an important role. Hence, the user should be able to translate, rotate and scale the 3D object. As technology progresses, the ability to represent highly complex objects increases. This arouses the desire to examine parts of the object in detail and also regarding the individual parts within a specified context. Additionally, the examination and analysis of the objects' inner structures can also be of interest. For example in the medical sector, having all the data digitally available to explore opens up plenty of possibilities. Hence, the challenge is how all these data can be provided with a maximum of interactivity for the user to be adapted optimally.\\
\newline
An idea is to let the user determine a region or a point of interest, regardless if this region is already visible, concealed by a jacket, or simply occluded by another object. In case it is not clearly visible, the jacket or the occluding object could simply be omitted, but this would also cause the context to get lost. As the context can be of significant importance, this approach is not eligible. In case the region or point of interest is already visible, the main purpose of the user can be to examine the individual parts of this region or point, or its connection and interaction with the remaining object. Again, without the context, the information can become less instructive. Hence, the idea is to reveal the region of interest whilst keeping the context.\\
\newline
Therefore, several scientific approaches exist. One idea is to simply lower the opacity of the outer structures. For example to reveal a brain, the approach would be to simply raise the transparency of the skull and the skin respectively for making the parts of secondary interest semi-transparent. This approach is called \emph{ghosting}. Another approach, \emph{cutaway views}, is to cut out parts of the object so that no information gets lost and still the region of interest would be fully visible. A different approach is a so-called \emph{explosion view} that unfolds an object by breaking it into multiple parts and shifting those apart whilst keeping the point of interest in the center.

\section{Motivation}
This thesis will examine several approaches to reveal regions of interest including inner structures of complex 3D models and at the same time retaining the context to be able to examine the region of interest in respect of its surroundings. Furthermore,\emph{VolumeShop}, an application for visualization research, is presented as a powerful tool to implement all sorts of approaches to support a maximum examination of complex 3D data.

\section{Problem statement and objectives}

