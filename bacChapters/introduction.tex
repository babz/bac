%contents:
%- motivation
%- problem statement
%- aim of the work
%- methodological approach
%- structure of the work


Data visualization is an substantial part of computer graphics. To make the visualized data easily examinable, user interaction plays an important role. Hence, the user should be able to translate, rotate and scale the 3D object. As technology progresses, the ability to represent highly complex objects increases. This arouses the desire to examine parts of the object in detail and also regarding the individual parts within a specified context. Additionally, the examination and analysis of the objects' inner structures can also be of interest. For example in the medical sector, having all the data digitally available to explore opens up plenty of possibilities. Hence, the challenge is how all these data can be provided with a maximum of interactivity for the user to be adapted optimally.\\
\newline
An idea is to let the user determine a region or a object of interest, regardless if this region is already visible, concealed by a jacket, or simply occluded by another object. In case it is not clearly visible, the jacket or the occluding object could simply be omitted, but this would also cause the context to get lost. As the context can be of significant importance, this approach is not eligible. In case the region or object of interest is already visible, the main aim of the user can be to examine the individual parts of this region or object, or its connection and interaction with the remaining objects. Again, without the context, the information can become less instructive. Hence, the idea is to reveal the region of interest whilst keeping the context.\\
\newline
Illustrative visualisation simplifies the representation of complex data. Several scientific approaches exist for the illustrative visualization. One idea is to simply lower the opacity of the outer structures. For example to reveal a brain, the approach would be to simply raise the transparency of the skull and the skin respectively for making the parts of secondary interest semi-transparent. This approach is called \emph{ghosting}. Another approach, \emph{cutaway views}, is to cut out parts of the object so that no information gets lost and still the region of interest would be fully visible. A different approach are so-called \emph{exploded views} that unfold an object by breaking it into multiple parts and shifting those apart whilst keeping the object of interest in the center.
%TODO peter: these methods come from technical illustration, and in visualization they are simulated by the techniques you describe????

\section{Motivation}
This thesis examines several approaches to reveal regions of interest including inner structures of complex 3D models. While revealing the regions of interest, these approaches retain the context to be able to examine the region of interest in respect of its surroundings. Furthermore,\emph{VolumeShop}, an application for visualization research, is presented as a powerful tool to implement various approaches to support a convenient level of examination of complex 3D data.

\section{Problem statement and objectives}
%cumbersome - mühsam
The implementation of high-level tools and satisfying various individual needs for examining an object can be cumbersome and time consuming. Furthermore, with increasing complexity of an object the rendering time can rise significantly as well. Therefore, an approach is demanded that considers the need for user-interaction to define regions or objects of interest, and additionally has an invariant computing effort for the rendering, regardless of the complexity of the object.\\
\newline
In section~\ref{ch:analysis}, several existing approaches are being presented and analyzed. Furthermore, the operation with VolumeShop and its concept is introduced in section~\ref{ch:method}. A simple algorithm for revealing occluded objects is proposed in section~\ref{ch:impl}.
